\begin{myitemize}
\item{Get the board connected to Internet} 

To be able to get the board connected to the Internet, you need to use the laptop as a NAT box. This is done by enabling “Share internet connection via ethernet port” on the laptop. The (only) ethernet interface of the laptop is connected to the board. The connection to the internet is done through the wireless (which should not have 802.1x enabled).



On the side of the board, we configure the network adaptor in the file /etc/network/ interfaces as following:
    \begin{tcolorbox}
    \begin{minted}{c}
    auto eth0
    iface eth0 inet static
    address 192.168.2.2
    netmask 255.255.255.0 gateway 192.168.2.1 dns-nameservers 192.168.2.1
    \end{minted}
    \end{tcolorbox}

After adding these lines to the file, the network stack must be started: 

\begin{tcolorbox}
    \begin{minted}{c}
    service networking start
    \end{minted}
\end{tcolorbox}

(this will also be done automatically at boot time).


\item{Test the network by running:}


\begin{tcolorbox}
    \begin{minted}{c}
    apt-get update
    ping 8.8.8.8
    \end{minted}
\end{tcolorbox}


\item{Getting ssh to work}

A bit more conveniënt way to have a terminal connection is directly over the network with ssh. This makes the installation of ssh necessary on the board:


\begin{tcolorbox}
    \begin{minted}{c}
    apt-get install ssh
    \end{minted}
\end{tcolorbox}

Next , set the root password with the "passwd" command.
Now, from the NAT host (i.c. your laptop), it is possible to connect though ssh:


\begin{tcolorbox}
    \begin{minted}{c}
    ssh -l root 192.168.2.2 
    \end{minted}
\end{tcolorbox}

The output will be : 


    \begin{minted}{c}
    root@localhost:~#
    \end{minted}

\end{myitemize}

