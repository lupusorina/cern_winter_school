\subsection{First Interaction with LINUX on System on Chip}

Now, we will move on with running an operating system on the System on Chip. For this, we will do the following steps:
\begin{myitemize}
\item you disconnected the JTAG cable used previously to programme the board
\item the mini SD Card is inserted in its socket
\item find Jumper 1 and Jumper 2 (JP1 and JP2) and make sure they are in the following order:
\begin{minted}{c}
JP1 OFF 
JP2 ON
\end{minted}
\end{myitemize}


When previously steps are done, you will see the RGB LED blinking on the board. It is very
important to make these configurations before moving on to the next section of the lab.

\subsection{Further configurations}

The SD card contains the image of Ubuntu 12.04 and will boot on the board at every 
restart.

\subsubsection{Screen}

\subsubsection{File transfer between host computer and the board}

% Connect a screen to see that it works

% Open a terminal and communicate 

% Send a file through SSH

\subsection{Run the first application}

In this part, we will do all the procedures to blink the RGB LED that is available 
on the board, as well as to read the accelerometer values.

\subsection{Webserver}
Now, we will learn how to display the accelerometer values on an webserver running
on the host computer.

